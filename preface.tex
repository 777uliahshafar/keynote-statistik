\documentclass[main.tex]{subfiles}

\begin{document}
\begin{withoutheadline}
	%\begin{frame}
	%   % Schriftgröße verkleinern
	%   \small
	%   % Absatzabstand einstellen
	%   \setlength{\parskip}{0.75\baselineskip}
	%
	%   Dies ist die eLectures-Präsentationsvorlage der Virtuellen PH.
	%
	%   Die kommenden Folien in \LaTeX\ können und \textbf{\emph{sollen} Sie nach Wunsch adaptieren}, wir bitten Sie aber, sich alle genau anzusehen und durchzulesen! Denn:
	%
	%   Auf jeder Folie finden Sie Anregungen, Hilfestellungen und Hinweise, sowohl für die inhaltlich-didaktische Vorbereitung als auch die Präsentation online. Zumindest die vorhergehende Titelfolie sollte bitte auf jeden Fall in dieser Form Verwendung finden (Wiedererkennungswert!)
	%
	%   Bitte senden Sie Ihre überarbeitete PDF-Datei \textbf{vor dem gemeinsamen Testtermin} direkt an Ihre/n Co-Moderator/in. Er/sie bespricht sie mit Ihnen und lädt sie für die eLecture im Raum hoch. Der Kontakt zur Co-Moderation wird aktiv von dieser hergestellt.
	%
	%   \begin{block}{Zur didaktischen Planung der Stunde:}
	%       Eine eLecture-Stunde geht oft sehr schnell vorbei! Gehen Sie daher bitte von max. \SI{75}{\percent} der Zeit für Ihre Präsentation aus, der Rest ist meist durch Einführung und Vorstellung, Interaktion, Rückfragen und Abschluss belegt. Je interaktiver Sie die eLecture halten, desto besser: besprechen Sie sich mit Ihrer CoModeration!
	%   \end{block}
	%\end{frame}

	\begin{frame}
		%\maketitle
		% \maketitle funktioniert auch im Article-Modus
		\titlepage
	\end{frame}
\end{withoutheadline}

\miniframesoff
\begin{frame}[label=inhalt]{Pratinjau}
	\tableofcontents
\end{frame}
\miniframeson

\end{document}
